\documentclass[12pt]{article}

\setlength{\topmargin}{-.75in} \addtolength{\textheight}{2.00in}
\setlength{\oddsidemargin}{.00in} \addtolength{\textwidth}{.75in}

\usepackage{amsmath,color,graphicx}
\usepackage{minted}

\nofiles

\pagestyle{empty}

\setlength{\parindent}{0in}


\begin{document}

\noindent {\sc {\bf {\Large Quiz 1}}
            \hfill EML4930/6934, Python Programming, Fall 2017}
\bigskip

\noindent {\sc  {}
            \hfill {\large Name:}
             \hfill}
\bigskip
\bigskip

\textbf{Instructions:} Please answer the questions below. You are not allowed to  use any notes or calculator on this quiz. You are not allowed to work with your neighbor. 
\bigskip


%\begin{enumerate}
%\item Name one difference between Python 2 and Python 3.
%\item Show a properly syntax single line comment in Python. Show a bulk/multi-line comment in Python.
%\item Consider floor division in Python 3. What will 3//2 return? 
%\item Consider a list named x. What is the easiest way to call the last item in list x?
%\item How should you properly denote a code block in Python?
%\item Consider the following code:
%\begin{minted}
%{python}
%for i in range(100):
%    print(i)
%\end{minted}
%What will be the first value of i? What will be the last value of i?
%\item Consider the following code:
%\begin{minted}
%{python}
%x = [1, 2, 3, 4, 5]
%y = [i**2 for i in x]
%\end{minted}
%List the values in y.
%\item Define a function solve the quadratic equation
%\begin{equation}
%ax^2 + bx + c = 0
%\end{equation} where $a,b,c$ are known numbers that are the input to your function. The solution is obtained as
%\begin{equation}
%x_1 = \frac{-b + \sqrt{b^2 - 4ac}}{2a}, x_2 = \frac{-b + \sqrt{b^2 - 4ac}}{2a}
%\end{equation} and your function should return $x_1$ and $x_2$.
%\item What is the output of the following code
%\begin{minted}
%{python}
%x = 1
%def new_x():
%    x = 2
%    print(x)
%new_x()
%print(x)
%\end{minted}
%\item What is the name of the function within a class that runs automatically on each new instance of an object?
%\end{enumerate}

{\bf Problem 1.}(2 points.) Consider a list named x. What is the easiest way to call the last item in list x?

\mint{python}|x[-1]|

\bigskip
\bigskip
\bigskip
\bigskip


{\bf Problem 2.}(4 points.) Consider the following code:
\begin{minted}
{python}
for i in range(100):
    print(i)
\end{minted}
What will be the first value of i? 

0

\bigskip
\bigskip

What will be the last value of i?

99

\bigskip
\bigskip
\bigskip
\bigskip
\bigskip
\bigskip


{\bf Problem 3.}(4 points.) The following code will print x twice in Python.
\begin{minted}
{python}
x = 1
def new_x():
    x = 2
    print(x)
new_x()
print(x)
\end{minted}
What is the value of the first print of x when running new\_x()?

2

\bigskip
\bigskip
\bigskip

What does the final print(x) display?


1

\bigskip
\bigskip
\bigskip
\bigskip
\bigskip
\bigskip
\bigskip
\bigskip
\bigskip
\bigskip
\bigskip
\bigskip


\end{document}
