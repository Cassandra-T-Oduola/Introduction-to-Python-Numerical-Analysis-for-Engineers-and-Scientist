\documentclass[12pt]{article}

\setlength{\topmargin}{-.75in} \addtolength{\textheight}{2.00in}
\setlength{\oddsidemargin}{.00in} \addtolength{\textwidth}{.75in}

\usepackage{amsmath,color,graphicx}
\usepackage{minted}

\nofiles

\pagestyle{empty}

\setlength{\parindent}{0in}


\begin{document}

\noindent {\sc {\bf {\Large Final review}}
            \hfill EML4930/6934, Python Programming, Fall 2017}
\bigskip

\noindent {\sc  {}
            \hfill {\large Name:}
             \hfill}
\bigskip
\bigskip

\textbf{Instructions:} Please answer the questions below. You are not allowed to  use any notes or calculator on this Exam. You are not allowed to work with your neighbor. 
\bigskip


\begin{enumerate}
\item Name one difference between Python 2 and Python 3.
\item Show a properly syntax single line comment in Python. Show a bulk/multi-line comment in Python.
\item Consider floor division in Python 3. What will 3//2 return? 
\item Consider a list named x. What is the easiest way to call the last item in list x?
\item How should you properly denote a code block in Python?
\item Consider the following code:
\begin{minted}
{python}
for i in range(100):
    print(i)
\end{minted}
What will be the first value of i? What will be the last value of i?
\item Consider the following code:
\begin{minted}
{python}
x = [1, 2, 3, 4, 5]
y = [i**2 for i in x]
\end{minted}
List the values in y.
\item Define a function solve the quadratic equation
\begin{equation}
ax^2 + bx + c = 0
\end{equation} where $a,b,c$ are known numbers that are the input to your function. The solution is obtained as
\begin{equation}
x_1 = \frac{-b + \sqrt{b^2 - 4ac}}{2a}, x_2 = \frac{-b + \sqrt{b^2 - 4ac}}{2a}
\end{equation} and your function should return $x_1$ and $x_2$.
\item What is the output of the following code
\begin{minted}
{python}
x = 1
def new_x():
    x = 2
    print(x)
new_x()
print(x)
\end{minted}
\item What is the name of the function within a class that runs automatically on each new instance of an object?
\item Consider an object named a. What is one way to list all of the attributes (and methods) within the object a?
\item How would you take the square root of 63.1516 in Python? (if you use a library you must stat the correct import)
\item Consider a list name x that already contains a lot of information. Consider a list y = ['bob', 'loves Python', 87289]. How would you add y to the end of list x?
\item z is a high dimensional numpy array. How would you find the index location of the maximum value of z?
\item Code a:
\begin{minted}
{python}
import numpy as np
x = np.random.random(100000)
y = []
for i in x:
    y.append(2.0*x)
y = np.array(y)
\end{minted}

Code b:
\begin{minted}
{python}
import numpy as np
x = np.random.random(100000)
y = 2.0*x
\end{minted}

Code a and code b do the exact same thing. Which code will run faster? (a, b, or both will run the same speed) Why?
\item Your friend is new to Python and programming. He is running the following code:
\begin{minted}
{python}
from __future__ import division
import numpy as np
x = np.ones(10, dtype='int')
y = np.random.random(10)
for i in range(10):
    x[i] = y[i]/2.0
\end{minted}
but he keeps getting that x = array([0, 0, 0, 0, 0, 0, 0, 0, 0, 0]). Why is each item in x zero?
\item given 
x = np.array([[4.0, 2.0],[-2.0,3.0]])

How would you transpose x? 
\item Write out the values in z
\begin{minted}
{python}
x = np.array([[4.0, 2.0],
              [-2.0,3.0]])
y = np.array([[2.0, 1.0],
              [0.0, 1.0])
z = x*y
\end{minted}
\item Consider the following matrix multiplication
\begin{equation}
\begin{bmatrix}
2 & 3 \\
1 & 0 \\
\end{bmatrix}\begin{bmatrix}
5 & 2 \\
1 & 3 \\
\end{bmatrix}\begin{bmatrix}
1 & 0 \\
2 & 0 \\
\end{bmatrix} = z
\end{equation} Write out all the code (including imports) to solve for z in Python.
\item Plot the function 
\begin{equation}
y(x) = x^3 + 2x + 10.0
\end{equation} on the domain \begin{equation}
10 \leq x \leq 19
\end{equation} using Python. Include all necessary imported libraries. 
\item Define a function that will return \textit{A}, \textit{B}. \textit{A} and \textit{B} are variables passed to the function. If B is not specified, it will default to a value of 1.0.
\item Open readme.txt using Python and print the contents.
\item What is the difference between 
\begin{minted}
{python}
f = open(myFile)
\end{minted}
and
\begin{minted}
{python}
with open(myFile) as f:
\end{minted}
\item Why Does the following code produce x = array([0, 0, 1]) in  Python?
\begin{minted}
{python}
x = np.array((0,0,0))
for i in range(3):
    x[i] = i/2.0
\end{minted}
Fix this code such that x = array([0.0, 0.5, 1.0]) after the for loop.
\item Correct this code
\begin{minted}
{python}
for i in range(10):
    if i > 5
        break
\end{minted}
such that it no longer produces this SyntaxError
\begin{minted}
{python}
    if i > 5
            ^
SyntaxError: invalid syntax 
\end{minted}
\item Why would you use numpy.random.seed() when generating random numbers?
\item Your friend wanted to grab the First item in the numpy array \textit{x}. Consider his code:
\begin{minted}
{python}
x = np.random.random(100)
y = x(1) 
\end{minted}
produces the Following syntax error
\begin{minted}
{python}
      1 x = np.random.random(100)
      2 x = np.random.random(100)
----> 3 y = x(1)

TypeError: 'numpy.ndarray' object is not callable
\end{minted}
What two things is your friend missing? 
\item What does the function \textit{os.system} do? Name a potential example where \textit{os.system} could be useful in a Python program? 
\item What is the Difference between Built-In Python libraries like os and math and libraries like numpy, pandas, and matplotlib?
\item Consider
\mint{python}|a = [0.0, 1.0, ['hi', 'world'], 2.0, 'blue']|
what will
\mint{python}|print(len(a))|
return?
\end{enumerate}


\end{document}
